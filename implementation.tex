\chapter{Implementation}
\label{chapter:implementation}

This chapter addresses the main decisions adopted regarding the implementation of the IDChain, vanilla an CA-based systems. Therefore, the following sections cover the technologies that were used in the development process and the implementation details of each component.

\section{Adopted Technologies}

In this section we will present the chosen technologies to implement the proposed work, for each component of the architecture.

\subsection{DHT}

In a first instance, the chosen library to implement the DHT node was \textit{TomP2P}\footnote{http://tomp2p.net/} a implementation of Kademlia in Java.
The use of TomP2P was already decided before-hand, since the Global Registry component in the reThink project, was already partially implemented using it.
This restricted the implementation options from the get-go, since it wasn't possible to compare different \ac{DHT} implementations and check which one could be more easily extended.

We tried to modify the TomP2P source code, and implement TLS connection supports.
This endeavor didn't succeeded, mainly since the TomP2P code is tightly coupled to TCP and UDP connections.
Since TomP2P uses \textit{Netty}\footnote{https://netty.io/}, a Java NIO client-server \ac{NIO} framework, we tried to to change Netty instance calls to TLS, a try which revealed infeasible.

Therefore we decided to implement the mechanisms using another DHT system.
Several DHT implementations exist in different languages, but was necessary to pin down an implementation that had a modular architecture and could be easily extended.

The implementation that revealed to be the most modular was the one of \textit{Kad}\footnote{https://kadtools.github.io}library, for \textit{Node.js}\footnote{https://nodejs.org/} and based on Kademlia.
This implementation allows to easily extend the base DHT with custom transports, middleware, storage layers and message processors, which enabled us to build a custom transport with TLS mutual-authentication and custom verification rules.

The same DHT client was used for the three different mechanisms: vanilla, CA-based and IDChain.
Is possible to switch between the three different mechanisms by declaring in the configuration file which one we want to use.

\subsection{Blockchain}


\subsection{API}

\subsubsection{Application Server}

\subsubsection{Database}


\subsection{Management Application}



