\chapter{Introduction}
\label{chapter:introduction}

\section{Motivation}
\label{section:motivation}

Nowadays with the current Internet infrastructure and the provided services built on top of it, the old telecommunications operators based services, like voice telephony, are losing importance.

\ac{OTT} players, like Google and Skype, are dominating the communications market with a no additional cost and closed ecosystem solutions.
New users will choose to use the services, that are used by the majority of their social environment.

\ac{OTT} services, by working in the closed ecosystem don't need to work in interoperability between services and communications standards, allowing them to be more competitive, agile and lead communication and multimedia innovation.
This could be problematic since it causes vendor lock-in and limits the portability of user identity and data, hinders innovation and block new entrants.

On the other hand, we have the worldwide Telco ecosystem that provide an highly reliable service and strong trustful identity.
Since is necessary to achieve worldwide service interoperability, the services provided rely on well-defined standards. These standards need to be agreed upon and defined, increasing the time to market of potential new services.
Telcos are also geographically restricted, so the deployment of new worldwide services could not be possible without roaming agreements in-place, which severely restricts Telcos in driving innovation.

The reThink~\footnote{https://rethink-project.eu/} project goal is to design a new peer-to-peer network infrastructure for communications based on Web technologies, that allow dynamic trusted relationships between distributed apps and a portable identity model, leveraging the advantages of the federated Telco and \ac{OTT} model.
The main goal is to achieve a framework that enable developers to create communication-based applications, allowing users from different reThink-based applications.

This dynamic trusted relationship is created by using \ac{Hyperties}, a web microservice paradigm, that enable the execution of trustful services in a web environment on user devices or network servers.
In order to achieve interoperability, the communication between hyperties is based on the \textit{Protocol-On-The-Fly}~\cite{protofly} concept, that allows using standard network protocols through a common API enabling communication between different hyperties from different service providers.
The hyperty concept allows to extend the communications beyond normal telephony and messaging, where even services using \ac{M2M} and \ac{IoT} systems could be built.
The hyperties are maintained by \acp{SP} and are loaded to the users device.

One of the main components in the reTHINK architecture is the \textbf{Registry} service.
The Registry service is a key-value based directory service, that facilitates the management and lookup of hyperty instances running in users devices.

The Registry service must have the following requirements:
\begin{itemize}
	\item \textit{Provide fast query response time}, since it will be accessed when establishing communication;
	\item \textit{Scalable}, since it will be a worldwide deployed service;
	\item \textit{High availability}, is necessary for communication establishment;
	\item \textit{Data consistency}, the hyperties information must be always up up to date, so it is possible to start the communication setup.
\end{itemize}

The Registry service is sub-divided in three components: \textit{Global Registry}, \textit{Domain Registry} and \textit{Local Registry}.
The \textit{Global Registry} is a key-value store based on \ac{DHT} technology, that stores user identifiers in hyperty services, index by a \ac{GUID}.
It lists domain-dependent identities owned by a user in the reTHINK system.
By resolving a \ac{GUID}, is possible to obtain these domain-dependent identities of each user.
The \textit{Domain Registry} is run by \acp{SP} and allows to lookup users hyperty instances information using the domain-dependent identities mentioned above.
It uses a client-server model, which allows to handle a high data update rate.
The \textit{Local Registry} component runs in the device runtime, that manages hyperty instances running in the runtime and contact the Global and Domain Registries.

\section{Problem Statement}
\label{section:problem-statement}

The \textit{Global Registry} is based on a \ac{DHT}, a \ac{P2P} distributed system, that usually is used in public-facing services, and therefore is open to participation, allowing any node to join the network.
One example, is the Mainline \ac{DHT}, that powers the BitTorrent network.

Several private and enterprise uses of DHT system exist, as the Dynamo\ref{dynamo2007} key-value system from Amazon, or the Cassandra\ref{cassandra2010} database.
These systems use \acp{DHT} as a building-block of their architecture for tasks that require multi-node coordination, as data sharding, for example, therefore not exposing the \ac{DHT} directly.
These solutions are also closed-source and don't have their source code available publicly.

The reTHINK framework is also inherently a federated solution, therefore isn't open to participation from outside the \acp{SP} organization, but it should also allow easily the entry of new nodes from the \acp{SP}.
This is an important aspect, because contrary to a private solution where we have full control of all the infrastructure, in a federated solution we don't have full control and is necessary to have additional infrastructure to create trust relations and verify identities between the \acp{SP} \ac{DHT} nodes.

\section{Proposed Solution}
\label{section:proposed}

The major goal of this thesis is to ensure the security of the \ac{DHT} necessary to build the Global Registry system.

In order to achieve this, a set of objectives need to be achieved:

\begin{itemize}
  \item \textbf{Mitigate common \ac{DHT} attacks}- several different type of attacks targetting the \ac{DHT} exist. A in-depth analysis of each one of them and existing solutions is necessary;
	\item \textbf{Guarantee communication privacy, integrity and authenticity}- it is necessary to secure the \ac{DHT} nodes communication;
	\item \textbf{Close DHT participation}- provide mechanisms to allow access-control nodes in the \ac{DHT};
	\item \textbf{Deal with nodes compromise}- allow to easily revoke access to compromised nodes or identities;
	\item \textbf{Minimize trust between participants}- even thought, our system could not tolerate maleficent \ac{SP} nodes, we want to minimize the trust between \ac{SP} nodes, so isn't possible to a couuple of \acp{SP} to monopolize the DHT system;
  \item \textbf{Maintain system decentralized}- solve all the aforementioned problems, maintaining a decentralized architecture, which should also minimize trust between \acp{SP}.
	\item \textbf{Isolate the DHT system from the Global Registy}- even though, the main use case of the DHT is the Global Registry, we want to decouple the DHT, by providing simple API that allow any application to build on top of.
\end{itemize}

We will build one DHT with two interchangeable mechanism that enable us to achieve the aforementioned objectives.

The first solution is based on more classic approach to establishing secure communications by using a \ac{CA} to issue certificates, enabling us to perform access-control to \ac{DHT} nodes and provide secure communications between nodes.

The second solution we will present is a novel approach that uses blockchain technology, which we named IDChain.
Using the blockchain we will build a \ac{DPKI} that will enable us to issue unique identities and certificates to nodes across all \acp{SP}.
This solution provides the same guarantees of the previous presented solution, and also allows maintaining a decentralized system and minimize the trust between participants at the same time.

This two solutions will have as basis a \ac{DHT} which doesn't provide secure communications between nodes nor access-control.
This is presented as a complementary mechanism, that we will use to benchmark our two security mechanisms.

\section{Outline}
This document describes the research and work developed and it is organized as follows:

\begin{itemize}
  \item \textbf{Chapter~\ref{chapter:introduction}} presents the motivation, background and proposed solution.
  \item \textbf{Chapter~\ref{chapter:relatedwork}} describes the state of the art of the technologies used in the solution architecture.
  \item \textbf{Chapter~\ref{chapter:architecture}} describes the system requirements and the architecture of the solution.
  \item \textbf{Chapter~\ref{chapter:implementation}} describes the implementation of the solution and the technologies chosen.
  \item \textbf{Chapter~\ref{chapter:evaluation}} describes the evaluation tests performed and the corresponding results.
  \item \textbf{Chapter~\ref{chapter:conclusion}} summarizes the work developed and future work.
\end{itemize}

\cleardoublepage
